\chapter{Background}

\section{Motivating Automated Parallelism Given the State of Modern
Computational Tools}
Moore's law dictates that the number of computational components per unit space
(such as an integrated circuit) doubles per unit time. These components cannot
necessarily be used to accelerate the rate of sequential computation, and
instead, the decrease in component size has been utilized by developing
multicore hardware. Further spatial limitations prevent shared memory
arrangements for large numbers of processors greater than around about 64 for
standard CPUs and about 1000 for new ``manycore'' processors \cite{manycore}.
This results in the need for distributed memory arrangements that rely on
high-latency communication between memory that cannot be written to directly by
any one processor. These distributed memory units are connected through a
network of connections that use one or more of many protocols like Ethernet or
Infiniband. These protocols vary in their attention to different factors,
including reliability, speed, ease of use, and level of abstraction.

Distributed memory hardware configurations, referred to as clusters, involve the
arrangement of many computers, referred to as nodes, that have their memory
interconnected by some communication protocol. They may contain over 1 million
nodes \cite{top500}. Due to the plethora of nodes, computation time within
individual nodes decreases relative to the communication time required to send
data between nodes. Because of this new bottleneck, developing
communication-mitigating algorithms has become a popular area of interest. Some
work involves modifying parallel algorithms such that the resulting algorithm
requires less communication \cite{strassen_comm_opt}. Other work involves
``pipelining'' algorithms such that communication and computation can overlap in
time \-- the program dedicates some threads within a node to performing
communication, while other threads carry out any computation that does not rely
on unreceived data \cite{gmres_pipe}.  More generally, methods that reduce the
need for synchronization between nodes can improve parallel CPU utilization.

Using algebraic and combinatorial code analysis, optimization, and generation
techniques, methods like polyhedral optimization and Decoupled Software
Pipelining (DSWP) have achieved automation of pipelining (and more generally the
parallelization of loops with data dependencies between iterations)
\cite{polyopt} \cite{dswp}. The development and use of these methods stems from
an implicit yet nonessential sequential ordering in the implementation of the
serial version of the algorithms upon which these methods operate. These
nonessential orderings derive from the procedural characteristics of the
languages used to implement the algorithms in question.

\section{Programming Models and Implicit Execution Order}
Programming languages emerged from abstract models of computation like lambda
calculus and Turing machines. Different ways of executing the computations
prescribed by these models gives way to different programming models and thus,
different ways of thinking about algorithm design. Naturally, these models have
different implicit assumptions, different flexibilities, and different
implications with regard to execution order. The fallout presents different
conditions and avenues for parallelism between two primary programming
paradigms, procedural and functional.

\subsection{Procedural}
Procedural programming has the propensity to overdetermine the behavior of a
program in such a way that renders it hard to parallelize. Namely, by
guaranteeing that each line of a program executes after the previous one has
completed, lines which don't explicitly depend on each other become
unnecessarily ordered. This guarantee does have upsides though, including the
fact that a programmer can easily and naturally encode execution dependencies:
lines which do in fact depend on previous lines will execute correctly, and
lines that depend on lines that appear afterward will fail, usually through a
compile-time or run-time error. But, in a parallel setting, a compiler,
interpreter, or scheduler will struggle to identify the full set of lines which
can execute in parallel with respect to each other. Depending on the language,
different difficulties arise; in languages with pointers or references, these
often come in identifying side effects of impure functions. In C++, for example,
pointer aliasing creates lots of havoc during any sort of code analysis, because
it allows the mutation of a single piece of data from many contexts.

\subsection{Functional}
In functional programming, a different implicit execution ordering exists.
Execution ordering derives from the parameterization and structure of function
calls, which imply the set of dependencies that each function has. Namely, a
fucution depends on any values that are passed to it. Any expressions passed as
arguments must be evaluated before the function accepting those arguments is
evaluated, or else the function cannot use those values during its own
computation. In truth, the function may not use a value that it takes as an
argument, so this style of implicit execution ordering also relies on the
programmer to properly specify the parameters to a given function. If parameters
are overspecified or specified in some inefficient way, then the compiler,
interpreter, or scheduler will likely assume that more dependencies exist than
in reality; such an overspecification hides parallelism. On the other hand,
underspecification takes on a simpler form in functional programming than in
procedural programming, and manifests simply as undeclared or unaccessible
variables in the current scope. Effectively, functional programming has a more
rigid notion of dependnecy and scope; this inflexibility, while arguably
semantically limiting in some ways, results in a more predictable and
potentially less error-prone model.

\subsection{Comparing Optimization}
The process of optimizing code in both procedural and functional languages takes
very different forms due to the differences above. Both languages optimize by
trying to find more computationally efficient yet equivalent forms of
expressions (or statements) present in the code, and by removing redundant or
unnecessary computations. But, procedural languages, due to their imprecise
dependency specifications, are sometimes optimized by reordering the evaluation
of certain expressions. Such optimizations make no sense in a functional world,
since no explicit ordering exists except for dependency-based orderings, which
cannot be changed without changing the meaning and possibly the outcome of the
code. Optimizing functional programming languages often takes the form of
deciding exactly when to evaluate blocks of code that have no dependence
relation. For example, consider a function $f(a,b)$ \-- $f$ takes $a$ and $b$ as
parameters, so depends on both; but, $a$ does not depend on $b$ nor vice-versa.
In order to avoid performing unnecessary computation, many functional languages
will try to determine where within $f$ (if at all) both $a$ and $b$ are used in
order to determine which should be computed first to maximize efficiency.
Functional languages that have lazy evaluation, such as Haskell, will actually
by default store these expressions as ``thunks'' that abide by ``call-by-need''
behavior: at runtime, thunks will evaluate only when needed, or directly used in
some computation. But, even Haskell's compiler, ghc, will try to look ahead to
see if it can guarantee that it will actually use the thunk's output. Eager
languages often perfrom the opposite optimization, looking into functions to see
if they actually don't need or will unlikely need the value, or if storing an
expression as a thunk will increase performance due to future partial function
applications or reuses of the function's value in different contexts. In pure
functional languages without mutability, functions always evaluate to the same
value so long as they have identical arguments, making common subexpression
elimination easier in certain cases and aiding in the elimination of redundant
computation.

\section{A History of Concurrency Control Constructs}
Preparing code for parallel execution shares similarities with optimizing code
for serial execution and critically hinges on the determination of execution
ordering and data dependencies. Filling in these blanks, due to the rationale
provided above, differs dramatically between procedural and functional
languages.

\subsection{Proceural Concurrency Control}
Procedural languages focus on managing data, or more generally, state: no two
processors should collide while reading or writing a piece of data. Semaphores
emerged as a fundamental means of providing this control \cite{semaphore},
behaving as blocking counters that allow only so many threads to move beyond
``lift'' calls simultaneously before other threads that have already done so
move beyond the symmetric ``lower'' calls on that semaphore. This incredibly
flexible solution proved complicated: the lift $P(S)$ for a sempahore $S$ calls
could occur anywhere, and so could the lower $V(S)$ calls. With an arbitary
number of threads in a language such as C executing arbitrary code and accessing
arbitrary data, numerous complicated errors can potentially arise, all various
forms of resource mismanagement and sources of errors or inefficiency: deadlock,
livelock, starvation, priority inversion, busy waiting, and others. Semaphores,
originally intended to provide a mechanism for controlling the scheduling of
computation assigned to different threads relative to each other, and serving as
a means of mutual exclusion, required very careful use to even provide this.
Race conditions arise whenever a thread writes a piece of data while another
thread simultaneously tries to either read or write it; semaphores contain no
indication of a thread's behavior nor precisely which data they govern. The
correctness and efficiency, and thus overall effectiveness, of their use depends
entirely upon the programmer.

Advancements came in the form of specializing control mechanisms for different
tasks. Mutexes better embodied the concept of mutual exclusion, requiring the
thread that locks (the corresponding ``lift'' operation) a mutex to also unlock
(``lower'') it. While more restrictive than semaphores, mutexes address a use
case where, truly, only one thread should be accessing some data at any given
point. Mutexes specialized further, gaining read-locks and read-write locks,
which allow for data to be read simultaneously by any number of processors but
written by only one (and ensuring that, while writing, no other processors can
read). But, mutexes fall prey to many of the same issues that semaphores do,
including the slew of deadlock variants.

Further parallelism control techniques came. One general category involves
atomic operations, which stemmed originally from hardware architectures and
their corresponding instruction sets. Read-modify-write nstructions such as
test-and-set and compare-and-swap allow for robust implementations of the above
control mechanisms, as well as more complex and fine-tuned non-blocking
(lock-free and wait-free) algorithms, usually implemented in the general case of
some data structure like a queue. These algorithms have only recently begun to
have practical performance characteristics when compared to blocking algorithms
\cite{waitfree_queue}. Data parallelism stems from identifying data structures
that can be operated on in an inherently mutually exclusive way, and while they
achieve a nearly ideal level of parallelism, they have restrictive use cases.

\subsection{Functional Concurrency Control}
In pure functional languages, any function calls with no relative dependencies
can safely execute in parallel, because they may not access the same data.
Functional programming languages pivot upon computation, not state: because they
lack mutable state, they also lack the ability to write a value while
simultaneously reading or writing it from another process. Lacking mutable state
yields correctness at the cost of efficiency. The programmer must now think
critically about how to avoid overspecifying function parameters such that two
functions that could run theoretically at the same time actually may.
Furthermore, runtime information can complicate this process: two threads that
want to modify elements of an array may not actually contend for the same data,
but the programmer, compiler, interpreter, or scheduler may not have the
capacity to identify this or react to this at runtime in a way that results
better performance. Ulimately, a purely functional language must either have a
very conservative but very safe means of parallelizing a program, or a more
aggresive but less safe means that may result in either confusing results or
retroactive inefficiencies.

Transactional memory serves as an example of the latter case: blocks of code are
marked as ``atomic'', and code executes in parallel with no explicit concurrency
control. If code within an atomic block interleaves with other code in a harmful
way, then the block (called a ``transaction'' in this context) restarts from the
beginning, discarding all modifications made since. Clojure has an advanced STM
system that allows the user to specify certain equivalences, allowing the STM
system to complete even if the prescribed ordering does not match the execution
ordering exactly. A user can specify that a function application commutes in time
with itself -- for example, a user can specify that an increment operation on a
counter commutes, so that any number of increments, in any order, performed in
transactions on that particular value will not cause transactions to restart.. Further
flexibility comes in the form of an \texttt{ensure} operation that Clojure uses to
require that a read value has not been mutated by that point. Clojure uses the
\texttt{alter} and \texttt{ref-set} operations as its primary read-write and write
operations within transactions. Finally, Clojure uses ``agents'' to perform
asynchronous operations on values without guaranteeing when other threads will
see those writes. Agents have their own set of functions that provide
flexibility: \texttt{await} blocks waiting for an agent, \texttt{set-validator}
associates some kind of check with the mutation of agents, \texttt{add-watch}
registers observers to trigger when an agent mutates, and \texttt{deref} and
\texttt{send} allow for read and read/write operations to interact properly with
those possibilities.

Clojure has a smorgasbord of other concurrency control mechanisms, ranging from
``reducers'' (which both amount to generalized, parallel SIMD operations) to
``transducers'' to ``futures'' and more.

\section{Leveraging Immutability for Parallelism}
\definecolor{inner}{HTML}{C1D1D4}
\definecolor{proc1}{HTML}{92BAA7}
\definecolor{proc2}{HTML}{7E7A9E}
\definecolor{proc3}{HTML}{E3CB30}
\tikzset
{
    treenode/.style = {
        text width = 1em, align = center, rectangle split,
        rectangle split horizontal,
        rectangle split parts = 2,
        rectangle split draw splits = true,
        rectangle split part align = base,
        draw, fill = inner
    },
    subtree/.style = {
        isosceles triangle, draw=black, align=center,
        minimum height=0.5cm, minimum width=1cm,
        shape border rotate=90, anchor=north
    },
    execute at begin node=\strut
}

\begin{tikzpicture}[->, >=stealth',
                    level/.style = {
                        sibling distance = 3cm/#1,
                        level distance = 2cm},
                    transform shape]

    \node [name=bpv1, treenode, fill=proc1] {* \nodepart{two} *}
        child {
            node [treenode, fill=proc1] {* \nodepart{two} *}
                child { node [name=bpv1clcl, treenode, fill=proc1] {1 \nodepart{two} 2} }
                child { node [treenode, fill=proc1] {3 \nodepart{two} 4} }
        }
        child {
            node [name=bpv1cr, treenode, fill=proc1] {* \nodepart{two} *}
                child { node [treenode, fill=proc1] {5 \nodepart{two} 6} }
                child { node [treenode, fill=proc1] {7 \nodepart{two} \_} }
        };

        \node [name=bpv2, treenode, right=4cm of bpv1, fill=proc2,
                label={\small \texttt{ind=3,Op(+,4)}}] {* \nodepart{two} *}
        child {
            node [name=bpv2cl, treenode, fill=proc2] {* \nodepart{two} *}
                child { node [draw=none] {} edge from parent[draw=none] }
                child {
                    node [treenode, fill=proc2] {3 \nodepart{two} 8}
                }
        }
        child { node [draw=none] {} edge from parent[draw=none] };

        \node [name=bpv3, treenode, right=2cm of bpv2, fill=proc3,
                label={\small \texttt{ind=3,Op(+,4)}}] {* \nodepart{two} *}
        child {
            node [name=bpv3cl, treenode, fill=proc3] {* \nodepart{two} *}
                child { node [draw=none] {} edge from parent[draw=none] }
                child {
                    node [treenode, fill=proc3] {3 \nodepart{two} 8}
                }
        }
        child { node [draw=none] {} edge from parent[draw=none] };

    \draw [-] (bpv2.two south) edge[out=240,in=60,dashed,->] (bpv1cr.north);
    \draw [-] (bpv3.two south) edge[out=240,in=60,dashed,->] (bpv1cr.north);
    \draw [-] (bpv2cl.one south) edge[out=240,in=60,dashed,->] (bpv1clcl.north);
    \draw [-] (bpv3cl.one south) edge[out=240,in=60,dashed,->] (bpv1clcl.north);
    %\draw [-latex] (test2.one south) -- (test3.one north);

\end{tikzpicture}

Clojure serves as a good example for how immutability can assist in automated
parallelism. It provides persistent data structures that appear immutable from
any reference, but can also accrue updates from those references, updates that
eventually propagate to those references. It does this without making entire
copies of those data structures, and instead by using pointers to catalog to
portions of the data structure that have not changed from any reference. An
excellent informal description of how persistent vectors work can be found at
Niklas D'orange's website \cite{persvec1}, and a detailed description of an
optimized persistent vector can be found in the ICPF 2015 conference proceedings
\cite{rrb_vec}.  Clojure’s persistent vector uses a b-trie for its internal
representation. The values in the vector are stored in leaves. If some reference
modifies the vector, it identifies the leaf of the trie that should contain the
modify value, creates a new version of that leaf with the updated value, and
creates a new root (and hence, a new trie) with the structure that results in
the modified vector without having to duplicate anything except the other parts
of the modified leaf. Sometimes new leaves (and the corresponding internal nodes
and branches to that leaf) must be created as well, and sometimes empty leaves
and internal nodes and branches must be pruned.  Clojure uses 32 children per
node, and achieves $O(\log_{32} n$) (what they consider effectively $O(1)$)
time for \texttt{append}, \texttt{update}, \texttt{lookup}, and \texttt{subvec},
as a mutable vector does. Even with this generous reading ($\log_{32} n \leq c$
for some constant upper-bound $c$), the persistent version does incur a constant
overhead involved with creating this representation of the vector and with
performing the maintenance of this trie. Clojure uses numerous tricks to
optimize this data structure. First, it uses bit partitioning, where the trie
branches based on 5 possible bit values (generating 32 branches per node). By
doing this, Clojure provides a means for looking up elements in a vector tied to
its internal representation. Looking at 5 bits at a time, the algorithm for
lookup walks through the internal nodes of the trie until it reaches a leaf,
where the value is stored. It furthermore uses a tail node, pointed to directly
by the root, to optimize modifications near the end of the vector.

\subsection{Transient Vectors: Relaxing Immutability When Helpful}
Transient versions of persistent vectors modify values directly. Persistent
vectors can be copy-constructed to transient vectors, and transient vectors that
have been modified are invalidated such that using them after modification
results in undefined behavior. Nodes and leaves are given IDs that correspond
to the reference that created them, and the transient vector can safely mutate
these. Finally, transient vectors can be copy-constructed back to persistent
vectors, where that reference behaves as any other persistent vector would,
invalidating the transient from which it was copied. By doing this, a stream of
unobserved mutations can take place without sacrificing the guise of
immutability. D'orange's masters thesis discusses optimizations of Relaxed Radix
Balanced Trees (RRB-Trees) through transience \cite{lorange2014rrb}.

\subsection{Persistency to Automate Concurrency Control without Race Conditions}
In a parallel application, processors can read and write to shared data. Before
performing a read of some data, a processor may want an updated version of that
data, or a version that corresponds to a certain point in a global ordering of
events, even if it stores that data in a persistent data structure. It may
hypothesize or expect that the data has changed and specifically want an updated
value. When this happens, it has to verify that the other process  has completed
its operations. Consider that each process converts an initial persistent vector
into a transient vector, where each process's vector gets its own ID. Then, that
process's transient vector sees its own modifications transiently, in real time,
but sees a persistent, immutable copy of the values other processes may have
modified. With the knowledge of both the values a process does not own and the
values it needs to read, any given process can safely obtain the most updated
version of a value: by reading the values it owns as it wishes, and by waiting
for other processes only when necessary. If multiple processes write to the same
locations in the vector, the situation becomes more complicated, and the values
must somehow be reconciled.

\begin{comment}
\subsection{Distributed memory parallelism}

Large numbers of cores executing in parallel cannot all share the same memory
due to physical limitations. More than

\subsection{Different categories of parallelism by methodology}
Preemptive
Postemptive

ideas for what else to talk about:

higher level models
  actors (erlang, etc)
  parallel stream processing
  map/fold (sorta covered already)
  STM
  message passing: MPI, zeromq, nanomsg, etc
  promises/futures (stemming from data dependency control)

task stealing (mercury) -- hey that was my idea!
  it's basically a sub-scheduler
  notion of "filling serial holes"
  relation to pipelining

the goal is to scoop out the theoretical model from the implementation
  should not be tied to syntax/semantics

TODO:
Cats by memory model
Shared
Distributed
Cloud
Persistent (careful with lingo here)
  git
  journaled HD
  databases

index stuff
tree-following (matt's dependency graph)

\end{comment}

