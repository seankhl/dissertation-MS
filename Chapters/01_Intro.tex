\chapter{Introduction}
Sometimes, data and their corresponding algorithms have a ``trivially
parallelizable'' characteristic (e.g. domain decomposition of the region of
interest in a simulation), where processors have, and write to, disjoint sets of
data. Even in this seemingly trivial case, processors may have to read data
produced by or shared with other processors and use that data as input to their
computations; or, various operations may require different, confounding sets of
data, requiring either data sharing across sets or a reconfiguration of those
sets. For example, matrix multiplications require a row/column ordering of
processors, but this does not minimize interprocessor boundaries as in an ideal
domain decomposition. Furthermore, matching these sets to other overlaid
structures for computation, such as meshes, presents further difficulties. Even
providing a balanced decomposition of data onto processors can itself prove
elusive and challenging, and exchange of data across the resulting boundaries may
require intricate and complex handling.

\section{Problem Description}
Most software in the molecular/quantum mechanics domain, including molecular
dynamics as well as implementations of spanning methods such as Hartree-Fock,
semi-empirical, density functional theory, continuum models like force field
models, and Monte Carlo, rely on legacy libraries to implement shared and
distributed memory parallelism \cite{orca} \cite{lammps}
\cite{lammps_montecarlo}. These libraries almost exclusively use OpenMP and MPI
for these respective tasks, sometimes even resorting to fine-grained and manual
shared resource management such as locks. In addition, almost all such software
uses BLAS/LAPACK and is written in FORTRAN, C, or C++ \cite{lapack} \cite{eigen}. Recently,
Python's SciPy implementations have surfaced, often in the form of high-level
and expressive wrappers that organize calls to low-level, optimized code
\cite{scipy}. GPU implementations of crucial SIMD operations have been done
manually and almost exclusively in CUDA for Nvidia GPUs and OpenCL for AMD GPUs
\cite{gpu}. More current technologies for distributed memory management in the
context of concurrency and parallelism have not integrated into the most popular
software, for historical but practical reasons: issues of compatibility,
maintainability, scalability, reliability, complexity, efficiency, and more have
prevented an elegant transition \cite{kokkos_port}. In specialized settings,
implementations of such algorithms have emerged \cite{erlang_hpc}
\cite{clojure_hpc}. But, this work often lacks comparisons to traditional
programs in terms of performance metrics and remains specialized and
disconnected from practice and end users. Furthermore, these implementations
often do not take advantage of the touted feature-sets of their languages or
environments, such as algorithmic immutability (realized as persistency in
Clojure), lazy evaluation, a functional programming perspective of state, and
abstraction. Books, manuals, and tutorials will recommend using an exposed
low-level interface to achieve higher performance, defeating the purpose of
using a higher-level language or environment. Haskell suggests writing code in
its internal representation to avoid situations where its compiler cannot
appropriately optimize code \cite{haskell_opt}.

In order to mitigate these problems, to bridge the gap between performance and
ease of programming, I strive to develop a model of concurrency control and
parallelism that can
\begin{enumerate}
    \item Allow programmers to implement algorithms in a way that faithfully
        describes the task at hand, without the model distracting from the
        clarity of that algorithm or requiring the programmer to redesign the
        algorithms to focus on implementation details;
    \item Allow programmers to leverage parallelism without requiring invasive
        changes to the algorithms that they parallelize, even when starting from
        optimized serial implementations that already exist;
    \item Operate efficiently and compete with the current state-of-the-art.
\end{enumerate}

When new theoretical improvements emerge in high performance computing
algorithms, programmers should not feel limited by the tools they have in
implementing, testing, benchmarking, and analyzing those algorithms in
real-world situations. Scientists should have tools that let them express their
ideas clearly and readily while also taking advantage of modern hardware and
software.

As a field, computer science still has a long way to go before the development
and existence of a ``holy grail'' solution where any program can immediately be
run in parallel in a truly optimal way. But, some itemized problems can be
addressed. Data races have emerged as a problem for both shared and distributed
memory parallelism in settings where multiple cores write and read the same
data, necessitating synchronization. By leveraging ideas from parallel
programming, namely persistent and transient data structures, data races can be
avoided a priori, removing the requirement of  manual synchronizing the
aforementioned reads and writes. I will describe and sketch the implementation
of a parallel container (that can be realized as a vector, hash map, b-tree, or
any other such data structure) in C++ with the following characteristics:
\begin{enumerate}
    \item Automoatic synchronization of reads and writes in shared memory
        without data races, and a model for how the same can be done in a
        distributed memory system;
    \item Usable as a drop-in replacement for a C++ standard library vector for
        a restricted set of member functions and inner types and operations on
        those types;
    \item Asymptotically identical runtimes for the operations when used in
        serial and the corresponding expected improvements when used in
        parallel;
    \item An interface for arbitrarily partitioning the vector across
        processors;
    \item An interface for extending the set of inner types that can be used;
    \item A model for generalizing the set of inner types and operations on
        those types used.
\end{enumerate}

I will also provide the implementation of the following actual parallel
algorithms using an implementation of this container to provide examples
of how it will work, and I will show how they correspond to current
implementations that use manual concurrency control and parallelism techniques:
\begin{enumerate}
    \item A ``foreach'' operation that takes the vector and applies one
        operation to all items in the vector (either a unary operation or a
        binary operation with another supplied input), resulting in an output
        vector of the same size;
    \item A ``reduce'' operation that takes the vector and applies one operation
        to all items in the vector, resulting in a final output value;
    \item A finite differences scheme that uses the vector to store the values
        of the heat equation on a 2D mesh;
\end{enumerate}

Then, I will discuss the method's effectiveness given different inputs and
conditions. Finally, I will discuss future directions for this method of
automating parallelism and its viability in incrementially solving the problem
of automated parallelism in the truly general case.

